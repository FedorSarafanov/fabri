\documentclass[border=0.5mm,tikz]{standalone}

\usepackage{ifthen}
\usepackage{cmap}
% \usepackage[defaultsans]{droidsans}
% \renewcommand*\familydefault{\sfdefault} %% Only if the base font of the document is to be typewriter style
\usepackage[T2A]{fontenc}
\usepackage[utf8x]{inputenc}
\usepackage{amsmath,amssymb}
\usepackage{esdiff,esint}


\usepackage{pgfplots}
\usetikzlibrary{shapes, arrows}
% \pgfplotsset{compat=1.10}
\pgfplotsset{compat=newest}
\usepgfplotslibrary{fillbetween}
\usetikzlibrary{patterns,calc}
\usepackage[outline]{contour}

\usetikzlibrary{quotes,angles}
% Example:
%   \lineann[1]{30}{2}{$L_1$}
% \newcommand{\lineann}[4][0.5]{%
%     \begin{scope}[rotate=#2, blue,inner sep=2pt]
%         \draw[dashed, blue!40] (0,0) -- +(0,#1)
%             node [coordinate, near end] (a) {};
%         \draw[dashed, blue!40] (#3,0) -- +(0,#1)
%             node [coordinate, near end] (b) {};
%         \draw[|<->|] (a) -- node[fill=white] {#4} (b);
%     \end{scope}
% }
\begin{document}

\begin{tikzpicture}
\xdef\len{8}

\xdef\wo{3}
\xdef\coeff{0.9}
\xdef\Z{\len}
\xdef\Shift{10}
\xdef\AngleShift{4}
\pgfmathparse{\len+1.2}\let\xmaxx\pgfmathresult
\pgfmathparse{\wo*sqrt(1+(\coeff*\Z/\wo^2)^2)}\let\H\pgfmathresult
\pgfmathparse{atan((\H)/(\Z+\Shift))+\AngleShift}\let\angle\pgfmathresult

\pgfmathparse{tan(\angle)*(\Z+\Shift)+0}\let\ymaxx
% l длина
% n полупериодов
% lena длина полпериода в координатах рисунка
% n/l*180*x
\pgfmathresult

\pgfmathparse{sqrt((\Z+\Shift)^2+\H^2)}\let\Rad\pgfmathresult

\tikzset{
    gaussfill/.style={
        fill=magenta, opacity=0,
    },
    gaussdraw/.style={
        blue, dashed
    },    
    sin/.style={
        black, line width=1pt,
    },
    sinfill/.style={
        fill=magenta, opacity=0.15,
    },
    mirror/.style={
        blue, line width=1.5pt,
        fill=blue!30
    },
    diel/.style={
       % pattern=north east lines,
       draw=black, line width=1pt, fill=yellow, opacity=0.8,
    },
}


\begin{axis}[axis lines=middle,
            xlabel={$f$},
            ylabel={$P$},
            % hide y axis,
            scale=0.4,
            width=7cm,
            height=5cm,
            % enlargelimits,
            unit vector ratio*=4 1 1,
            xtick=\empty,
            ytick=\empty,
             extra x ticks={0}, % дополнительные метки на осях
            extra x tick labels={   % можно указать специальные подписи к ним
                    {$f_\text{рез}$},
                }, 
            % ymin=-\ymaxx,
            ymax=30,
            xmin=-6,
            xmax=6,
            % xmin=-23,
            % xmax=23,
            % xmax={\xmaxx},
            disabledatascaling,
            x axis line style = {dotted, black!50},
            y axis line style = {xshift=-77*0.4},
            x label style={
                at={(axis description cs:1.07,0)},
                anchor=center,      % расположение метки ровно в точке (1.1,0)
                rotate=360,         % вообще метку еще можно повернуть)
                black               % цвет метки
            },
            y label style={
                at={(axis description cs:0.09,0.95)},
                anchor=center,      % расположение метки ровно в точке (1.1,0)
                rotate=360,         % вообще метку еще можно повернуть)
                black               % цвет метки
            },
            % domain=-2:2, 
            % restrict y to domain=-2:2,
            % xtick={1,4},
            % xticklabels={a,b}
            ]


% \contourlength{0.6mm};
\addplot[name path=A,domain={-10:10},samples=150] {exp(-x^2)*30};% node[pos=.5, above]{$\Delta f_q$};

\addplot[domain={-10:10},samples=150] {exp(-(x/2)^2)*15};% node[pos=.5, above]{$\Delta f_q$};

\addplot[domain={-10:10},samples=150] {exp(-(x/1.5)^2)*20};% node[pos=.5, above]{$\Delta f_q$};

% \addplot[name path=A,domain={-15:-5},samples=150] {exp(-(x+10)^2)*25} node[pos=.5, above]{$\Delta f_{q-1}$};

% \addplot[name path=A,domain={5:15},samples=150] {exp(-(x-10)^2)*25} node[pos=.5, above]{$\Delta f_{q+1}$};

% \xdef\X{1}
% \pgfmathparse{exp(-(\X)^2)*30}\let\Y\pgfmathresult
% \draw[<-] (-\X,\Y) -- ++ (-2,0);
% \draw[<-] (\X,\Y) -- ++ (2,0);

% \draw[<-] ({-1},{30/exp(1)}) -- ++(-2,0);
% \draw[<-] ({+1},{30/exp(1)}) -- ++ (2,0);% node [right,yshift=0.5em, xshift=-0.5em] {$\Delta f_{q}$};

% \draw[<-] ({-11},{20/exp(1)}) -- ++(-2,0);% node [left,yshift=0.5em, xshift=0.5em] {$\Delta f_{q-1}$};
% \draw[<-] ({-9},{20/exp(1)}) -- ++ (2,0);

% \draw[<-] ({9},{20/exp(1)}) -- ++(-2,0);
% \draw[<-] ({11},{20/exp(1)}) -- ++ (2,0);% node [right,yshift=0.5em, xshift=-0.5em] {$\Delta f_{q+1}$};

% \xdef\X{10}
% \pgfmathparse{exp(-(\X-10)^2)*20}\let\Y\pgfmathresult
% \draw[<-] (-\X,\Y) -- ++ (-2,0);
% \draw[<-] (\X,\Y) -- ++ (2,0);


% \addplot[name path=B,gaussdraw,domain={-\len:\len},samples=100] {-\wo*sqrt(1+(\coeff*x/\wo^2)^2)} node[pos=.8, above]{};

% \addplot[gaussfill]fill between[of=A and B, soft clip={domain=-\len:\len}];


% \begin{scope}[xscale=-1]
% \draw[draw=none] (0,0) coordinate (0) -- (\Z,\H) coordinate (A);
% \draw[draw=none] (0,0) -- (\Z,-\H) coordinate (B);
% \path[mirror,draw=none,opacity=0,name path=C]([shift=(-\angle:\Rad)]-\Shift,0) arc (-\angle:\angle:\Rad);
% \draw[mirror] ([shift=(-\angle:\Rad)]-\Shift,0) coordinate (2) arc (-\angle:\angle:\Rad) -- ++ (1.3,0) -- ($(2)+(1.3,0)$) -- (2) -- cycle;
% \path[name path=D] (A) -- (B);
% \addplot[gaussfill]fill between[of=C and D];
% \end{scope}

% \begin{scope}[xscale=1]
% \draw[draw=none] (0,0) coordinate (0) -- (\Z,\H) coordinate (A);
% \draw[draw=none] (0,0) -- (\Z,-\H) coordinate (B);
% \path[mirror,draw=none,opacity=0,name path=C]([shift=(-\angle:\Rad)]-\Shift,0) arc (-\angle:\angle:\Rad);
% \draw[mirror] ([shift=(-\angle:\Rad)]-\Shift,0) coordinate (2) arc (-\angle:\angle:\Rad) -- ++ (1.3,0) -- ($(2)+(1.3,0)$) -- (2) -- cycle;
% \path[name path=D] (A) -- (B);
% \addplot[gaussfill]fill between[of=C and D];
% \end{scope}

% \draw[dashed] (-\Rad+\Shift,0) -- (\Rad-\Shift,0);

% \xdef\n{13}
% \xdef\inside{0}
% \xdef\scr{1} % сжатие по вертикали в диэлектрике
% \pgfmathparse{\n/(\Rad-\Shift)/2*180}\let\omegga\pgfmathresult
% \pgfmathparse{1/(\n/(\Rad-\Shift)/2)}\let\lena\pgfmathresult
% % l длина
% % n полупериодов
% % lena длина полпериода в координатах рисунка
% % n/l*180*x
% \ifthenelse{\isodd{\n}}%
%     {%нечет
%         \xdef\phhi{90}
%     }%чет
%     {  
%         \xdef\phhi{0}
%         % \xdef\lefft{\lena}
%     }%
%         \xdef\lefft{\lena*\inside/2}

% \pgfmathparse{\Rad-\Shift}\let\mirror\pgfmathresult
% % \draw[diel] (-\lefft,-\ymaxx) rectangle (\lefft,\ymaxx) ;

% % \draw (0,\ymaxx) node[below] {$n$};
% % \draw (0,-\ymaxx) node[below] {$m=\inside$};
% % \draw (0,0) node {\ymaxx};

% \draw[dashed, blue!40] (-\mirror,\ymaxx) -- +(0,2.5)
%             node [coordinate, near end] (a) {};
%         \draw[dashed, blue!40] (\mirror,\ymaxx) -- +(0,2.5)
%             node [coordinate, near end] (b) {};
%         \draw[|<->|] (a) -- node[fill=white] {$L$} (b);

% \addplot[name path=psinus1,sin,domain={-\Rad+\Shift:\Rad-\Shift},samples=400] {\wo*sqrt(1+(\coeff*x/\wo^2)^2)*sin(\omegga*x+\phhi)*(abs(x)<\lefft?\scr:1)} node[pos=.8, above]{};

% \addplot[name path=psinus2,sin,domain={-\Rad+\Shift:\Rad-\Shift},samples=400] {-\wo*sqrt(1+(\coeff*x/\wo^2)^2)*sin(\omegga*x+\phhi)*(abs(x)<\lefft?\scr:1)} node[pos=.8, above]{}; 

% \addplot[sinfill] fill between [of=psinus1 and psinus2, soft clip={domain=-\Rad+\Shift:\Rad-\Shift}];



\end{axis}
\end{tikzpicture}
\end{document}
